\documentclass[11pt,a4paper]{article}
\usepackage[utf8]{inputenc}
\usepackage[T1]{fontenc}
\usepackage[french]{babel} % Change to [english] for English reports
\usepackage{lmodern}
\usepackage{geometry}
\usepackage{xcolor}
\usepackage{tikz}
\usepackage{pgfplots}
\usepackage{booktabs}
\usepackage{array}
\usepackage{longtable}
\usepackage{fancyhdr}
\usepackage{titlesec}
\usepackage{enumitem}
\usepackage{multirow}

% ========================================
% TEMPLATE CONFIGURATION VARIABLES
% ========================================

% Participant Information
\newcommand{\participantName}{[PARTICIPANT_NAME]}
\newcommand{\participantAge}{[PARTICIPANT_AGE]}
\newcommand{\participantGender}{[PARTICIPANT_GENDER]}
\newcommand{\participantProfession}{[PARTICIPANT_PROFESSION]}
\newcommand{\evaluationDate}{[EVALUATION_DATE]}

% RAADS-R Scores
\newcommand{\totalScore}{[TOTAL_SCORE]}
\newcommand{\maxTotalScore}{240}
\newcommand{\socialScore}{[SOCIAL_SCORE]}
\newcommand{\maxSocialScore}{117}
\newcommand{\sensoryScore}{[SENSORY_SCORE]}
\newcommand{\maxSensoryScore}{60}
\newcommand{\restrictedScore}{[RESTRICTED_SCORE]}
\newcommand{\maxRestrictedScore}{42}
\newcommand{\languageScore}{[LANGUAGE_SCORE]}
\newcommand{\maxLanguageScore}{21}

% Interpretation
\newcommand{\interpretationLevel}{[INTERPRETATION_LEVEL]}
\newcommand{\interpretationDescription}{[INTERPRETATION_DESCRIPTION]}

% Language-specific labels (French by default)
\newcommand{\reportTitle}{RAPPORT D'ÉVALUATION}
\newcommand{\testName}{Test RAADS-R}
\newcommand{\testFullName}{Ritvo Autism Asperger Diagnostic Scale - Revised}
\newcommand{\participantLabel}{Participant :}
\newcommand{\ageLabel}{Âge :}
\newcommand{\genderLabel}{Genre :}
\newcommand{\professionLabel}{Profession :}
\newcommand{\evaluationDateLabel}{Date d'évaluation :}
\newcommand{\psychologistLabel}{Psychologue évaluateur :}
\newcommand{\psychologistTitle}{Spécialiste TSA adultes}

% ========================================

% Configuration de la page
\geometry{margin=2.5cm}
\pagestyle{fancy}
\fancyhf{}
\fancyhead[L]{\textcolor{primary}{Évaluation RAADS-R}}
\fancyhead[R]{\textcolor{primary}{\participantName\ - \participantAge\ ans}}
\fancyfoot[C]{\thepage}

% Couleurs personnalisées
\definecolor{primary}{RGB}{41, 128, 185}
\definecolor{secondary}{RGB}{52, 73, 94}
\definecolor{accent}{RGB}{231, 76, 60}
\definecolor{success}{RGB}{39, 174, 96}
\definecolor{warning}{RGB}{243, 156, 18}
\definecolor{lightgray}{RGB}{236, 240, 241}

% Style des titres
\titleformat{\section}{\Large\bfseries\color{primary}}{}{0em}{}[\titlerule]
\titleformat{\subsection}{\large\bfseries\color{secondary}}{}{0em}{}

% Configuration pgfplots
\pgfplotsset{compat=1.18}

\begin{document}

\begin{titlepage}
\centering
\vspace*{2cm}

{\Huge\bfseries\color{primary} \reportTitle}\\[0.5cm]
{\LARGE\color{secondary} \testName}\\[1cm]
{\Large \testFullName}\\[2cm]

\begin{tikzpicture}
\draw[primary, line width=3pt] (-4,0) -- (4,0);
\end{tikzpicture}\\[2cm]

{\Large\bfseries \participantLabel} {\Large \participantName}\\[0.5cm]
{\Large\bfseries \ageLabel} {\Large \participantAge\ ans}\\[0.5cm]
{\Large\bfseries \genderLabel} {\Large \participantGender}\\[0.5cm]
{\Large\bfseries \professionLabel} {\Large \participantProfession}\\[2cm]

{\Large\bfseries \evaluationDateLabel} {\Large \evaluationDate}\\[0.5cm]
{\Large\bfseries \psychologistLabel} {\Large \psychologistTitle}\\[3cm]

\vfill
{\color{secondary}\rule{\linewidth}{2pt}}
\end{titlepage}

\newpage

\section{Synthèse exécutive}

\begin{center}
\colorbox{accent!20}{\begin{minipage}{0.9\textwidth}
\centering
\vspace{0.5cm}
{\Large\bfseries\color{accent} RÉSULTAT PRINCIPAL}\\[0.5cm]
{\huge\bfseries Score total : \totalScore/\maxTotalScore}\\[0.3cm]
{\Large\bfseries\color{accent} \MakeUppercase{\interpretationLevel}}
\vspace{0.5cm}
\end{minipage}}
\end{center}

Le test RAADS-R (Ritvo Autism Asperger Diagnostic Scale-Revised) est un instrument d'auto-évaluation standardisé pour le diagnostic des troubles du spectre autistique chez les adultes. Avec un score de \totalScore, [INTERPRETATION_CONTEXT], les résultats de \participantName\ indiquent une \textbf{\interpretationDescription}.

\subsection{Répartition des scores par domaine}

\begin{center}
\begin{tikzpicture}
\begin{axis}[
    ybar,
    width=14cm,
    height=8cm,
    ylabel={Score obtenu},
    xlabel={Domaines évalués},
    ymin=0,
    ymax=120,
    xtick=data,
    xticklabels={Social, Sensoriel/Moteur, Intérêts restreints, Langage},
    x tick label style={rotate=45, anchor=east},
    bar width=1.5cm,
    nodes near coords,
    every node near coord/.append style={font=\Large\bfseries},
]
\addplot[fill=primary!70] coordinates {
    (1,\socialScore)
    (2,\sensoryScore)
    (3,\restrictedScore)
    (4,\languageScore)
};
\addplot[fill=lightgray!50, draw=secondary] coordinates {
    (1,\maxSocialScore)
    (2,\maxSensoryScore)
    (3,\maxRestrictedScore)
    (4,\maxLanguageScore)
};
\legend{Score obtenu, Score maximum}
\end{axis}
\end{tikzpicture}
\end{center}

\section{Analyse détaillée par domaine}

\subsection{Domaine Social (\socialScore/\maxSocialScore\ points)}

[SOCIAL_DOMAIN_ANALYSIS]

% Template sections for detailed analysis
\textbf{Compétences sociales fondamentales :}
\begin{itemize}[leftmargin=2cm]
\item [SOCIAL_SKILL_1]
\item [SOCIAL_SKILL_2]
\item [SOCIAL_SKILL_3]
\item [SOCIAL_SKILL_4]
\end{itemize}

\textbf{Communication interpersonnelle :}
\begin{itemize}[leftmargin=2cm]
\item [COMMUNICATION_1]
\item [COMMUNICATION_2]
\item [COMMUNICATION_3]
\item [COMMUNICATION_4]
\end{itemize}

\textbf{Relations interpersonnelles :}
\begin{itemize}[leftmargin=2cm]
\item [RELATIONSHIP_1]
\item [RELATIONSHIP_2]
\item [RELATIONSHIP_3]
\end{itemize}

\subsection{Domaine Sensoriel/Moteur (\sensoryScore/\maxSensoryScore\ points)}

[SENSORY_DOMAIN_ANALYSIS]

\textbf{Sensibilités sensorielles :}
\begin{itemize}[leftmargin=2cm]
\item [SENSORY_1]
\item [SENSORY_2]
\item [SENSORY_3]
\item [SENSORY_4]
\end{itemize}

\textbf{Régulation sensorielle :}
\begin{itemize}[leftmargin=2cm]
\item [REGULATION_1]
\item [REGULATION_2]
\item [REGULATION_3]
\end{itemize}

\textbf{Évolution temporelle :}
[SENSORY_EVOLUTION]

\subsection{Domaine Intérêts Restreints (\restrictedScore/\maxRestrictedScore\ points)}

[RESTRICTED_DOMAIN_ANALYSIS]

\textbf{Patterns d'intérêts spécialisés :}
\begin{itemize}[leftmargin=2cm]
\item [INTEREST_1]
\item [INTEREST_2]
\item [INTEREST_3]
\item [INTEREST_4]
\end{itemize}

\textbf{Communication et intérêts :}
\begin{itemize}[leftmargin=2cm]
\item [COMMUNICATION_INTEREST_1]
\item [COMMUNICATION_INTEREST_2]
\item [COMMUNICATION_INTEREST_3]
\end{itemize}

\subsection{Domaine Langage (\languageScore/\maxLanguageScore\ points)}

[LANGUAGE_DOMAIN_ANALYSIS]

\textbf{Compétences acquises :}
\begin{itemize}[leftmargin=2cm]
\item [LANGUAGE_STRENGTH_1]
\item [LANGUAGE_STRENGTH_2]
\item [LANGUAGE_STRENGTH_3]
\end{itemize}

\textbf{Quelques difficultés persistantes :}
\begin{itemize}[leftmargin=2cm]
\item [LANGUAGE_DIFFICULTY_1]
\item [LANGUAGE_DIFFICULTY_2]
\end{itemize}

\section{Analyse développementale}

\subsection{Évolution des symptômes}

[DEVELOPMENTAL_ANALYSIS]

\textbf{Persistance depuis l'enfance :}
\begin{itemize}[leftmargin=2cm]
\item [CHILDHOOD_PERSISTENCE_1]
\item [CHILDHOOD_PERSISTENCE_2]
\item [CHILDHOOD_PERSISTENCE_3]
\end{itemize}

\textbf{Développement à l'âge adulte :}
\begin{itemize}[leftmargin=2cm]
\item [ADULT_DEVELOPMENT_1]
\item [ADULT_DEVELOPMENT_2]
\end{itemize}

\textbf{Améliorations avec l'âge :}
\begin{itemize}[leftmargin=2cm]
\item [IMPROVEMENT_1]
\item [IMPROVEMENT_2]
\item [IMPROVEMENT_3]
\end{itemize}

\section{Facteurs de force et de compensation}

\textbf{Compétences développées :}
\begin{itemize}[leftmargin=2cm]
\item [STRENGTH_1]
\item [STRENGTH_2]
\item [STRENGTH_3]
\item [STRENGTH_4]
\end{itemize}

\textbf{Stratégies de compensation :}
\begin{itemize}[leftmargin=2cm]
\item [COMPENSATION_1]
\item [COMPENSATION_2]
\item [COMPENSATION_3]
\end{itemize}

\section{Recommandations}

\subsection{Évaluation diagnostique formelle}

[DIAGNOSTIC_RECOMMENDATION]

\begin{itemize}[leftmargin=2cm]
\item [DIAGNOSTIC_STEP_1]
\item [DIAGNOSTIC_STEP_2]
\item [DIAGNOSTIC_STEP_3]
\item [DIAGNOSTIC_STEP_4]
\end{itemize}

\subsection{Stratégies de soutien}

\textbf{Gestion sensorielle :}
\begin{itemize}[leftmargin=2cm]
\item [SENSORY_SUPPORT_1]
\item [SENSORY_SUPPORT_2]
\item [SENSORY_SUPPORT_3]
\end{itemize}

\textbf{Compétences sociales :}
\begin{itemize}[leftmargin=2cm]
\item [SOCIAL_SUPPORT_1]
\item [SOCIAL_SUPPORT_2]
\item [SOCIAL_SUPPORT_3]
\end{itemize}

\textbf{Valorisation des forces :}
\begin{itemize}[leftmargin=2cm]
\item [STRENGTH_SUPPORT_1]
\item [STRENGTH_SUPPORT_2]
\item [STRENGTH_SUPPORT_3]
\end{itemize}

\section{Conclusion}

[CONCLUSION_INTRO] \participantName\ (\totalScore/\maxTotalScore) [CONCLUSION_INTERPRETATION]. Le profil révèle :

\begin{itemize}[leftmargin=2cm]
\item [CONCLUSION_POINT_1]
\item [CONCLUSION_POINT_2]
\item [CONCLUSION_POINT_3]
\item [CONCLUSION_POINT_4]
\end{itemize}

[CONCLUSION_CLINICAL_INTERPRETATION]

[CONCLUSION_FINAL_RECOMMENDATION]

\vfill
\begin{center}
{\color{secondary}\rule{\linewidth}{1pt}}\\[0.3cm]
{\footnotesize Rapport compilé le \today\ en utilisant Claude AI}
\end{center}

\end{document}

% ========================================
% TEMPLATE USAGE INSTRUCTIONS
% ========================================
%
% To use this template:
%
% 1. Replace all variables in square brackets with actual values:
%    - [PARTICIPANT_NAME], [PARTICIPANT_AGE], etc.
%    - [TOTAL_SCORE], [SOCIAL_SCORE], etc.
%    - [INTERPRETATION_LEVEL], [INTERPRETATION_DESCRIPTION]
%
% 2. Replace content placeholders:
%    - [SOCIAL_DOMAIN_ANALYSIS] with detailed analysis
%    - [SOCIAL_SKILL_1] through [SOCIAL_SKILL_4] with specific observations
%    - Continue for all bracketed content areas
%
% 3. For different languages:
%    - Change \usepackage[french]{babel} to desired language
%    - Update language-specific labels in the configuration section
%
% 4. Color scheme can be modified by changing the RGB values in the
%    color definitions section
%
% 5. The chart will automatically update based on the score variables
%
% Example usage in code:
% - Find and replace [PARTICIPANT_NAME] with "John Doe"
% - Find and replace [TOTAL_SCORE] with "89"
% - Continue for all variables
%
% ========================================